\section{Symmetry, Conservation Laws, and Selection Rules}
\subsection{Operators in quantum theory}
\subsubsection{Fundamental opertors}
The fundamental operators in quantum theory are $x, p$ with 
\[ 
    (x\, \psi)(x) = x\psi(x), \quad (p\, \psi)(x) = -i\hb \pd x
\] 
They give rise to the canonical commutation relation 
\[ 
    [x, p] = i\hb 
\] 
This commutation relation gives rise to vector operator's commutation relations. 
The momentum operator may be conveniently memorized as using $-\pd x$ to effect 
translation in $x$, using $i$ to keep it skew-Hermitian, using $\hb$ for units. 

\subsubsection{Ladder operators}
For a harmonic oscillator Hamiltonian 
\[ 
    H = \df 1 {2m} \left[p^2 + (m\omega x)^2\right]
\] 
The raising and lowering opperators of interest are 
\[ 
    a_\pm = \df 1 {\sqrt{2\hb m \omega}}\left(\mp i p + m\omega x\right)
\] 
They are conjugates of each other and obey the commutation relations 
\[ 
    [a_-, a_+] = 1, \quad H = \hb \omega \left(a_+a_- + \df 1 2\right)
\] 
We can in turn express $x, p$ in terms of the ladder operators 
\[ 
    x = \sqrt{\df{\hb}{2m\omega}}(a_+ + a_-), \quad p = i \sqrt{\df{\hb m \omega}{2}}(a_+ + a_-)
\] 
Number operator $n$ is defined by $n = a_+a_-$. We conveniently have (operator on left side) 
\[ 
    n\psi_n = n\psi_n, \quad a_+\psi_n = \sqrt{n+1}\psi_{n+1}, \quad a_-\psi_n = \sqrt n \psi_{n-1}
\] 

\subsubsection{Vector operators}
The fundamental commutation relations for vector operators are 
\[ 
    [V_x, V_y] = i\hb V_z; \quad [V_y, V_z] = i\hb V_x; \quad [V_z, V_x] = i\hb V_y 
\] 
We can define $V^2 = V_x^2 + V_y^2 + V_z^2$, then $[V, V^2]=0$. 
We can also define ladder operators 
\[ 
    V_\pm = V_x \pm i V_y 
\] 
They obey the commutation relations 
\[ 
    [V^2, V_\pm] = 0, \quad [V_z, V_\pm] = \pm \hb V_\pm
\] 
Fix an eigenspace of $L^2$ with eigenvalue $\hb^2l(l+1)$ with 
dimension $2l+1$, where $l$ can be an integer or half-integer.
In this eigenspace, 
\[ 
    V_z = \hb \begin{pmatrix}
        l \\ & l-1 \\ &&\ddots \\ &&&-l 
    \end{pmatrix}, \quad
    L_x = \df 1 2 (V_+ + V_-), \quad 
    L_y = \df i {2}(V_- - V_+)
\] 
Here the representations of $V_\pm$ are shifted diagonals of $\hb$. 

\subsection{Transformation}
We first formally consider transformations acting on states. 
\begin{definition}[translation operator]
    We define its action on $\psi(x)$ as 
    \[T(a)\psi(x)=\psi(x-a)\]
    This definition characterizes the action of $T(a)$ on the positional 
    representation of $|\psi\ra$. In terms of states themselves, the translation 
    operator satisfy, for $|\psi'\ra =T(a)|\psi\ra$
    \[\la \psi'|x\ra = \la \psi|x-a\ra\]
\end{definition}
\begin{definition}[parity operator]
    The parity-inverted state $|\psi'\ra = \Pi |\psi\ra$ satisfies 
    \[\ang{\psi'|r} = \ang{\psi'|-r}\]
    Here $|r\ra$ may be generally understood as a 3d positional basis. 
    Represented spherically:
    \[\Pi \psi(r, \theta, \phi) = \psi(r, \pi - \theta, \phi + \pi)\]
\end{definition}
\begin{proposition} For hydrogenic orbitals, 
    $\Pi \psi_{nlm}(r, \theta, \phi) = (-1)^l\psi_{nlm}(r, \theta, \phi)$
    We only need consider the angular part. Recall
    $Y_l^m(\theta, \phi)\propto P_l^m(\cos\theta)\exp(im\phi)$, 
    and $\cos(\pi - \theta)=-\cos\theta$.
\end{proposition}

\begin{definition}[$z$-rotation operator]
    The rotation operator which effects counterclockwise rotation about the 
    $z$-axis by $\varphi$ is 
    \[ 
        R_z(\varphi)\psi(r, \theta, \phi)=\psi(r, \theta, \phi - \varphi)
    \] 
\end{definition}
Transformations, since they map states to states, must be unitary. 
They are effectively a basis permutation. For example, translation effects the 
basis change $|x\ra \mapsto |x-a\ra$. One may view this via the equivalence between 
sliding our state to the right, and sliding the spatial axis to the left while 
following the center of the axis. By unitarity $T^\dag = T^{-1}$. 

Apart from considering the effect of $T$ on $|\psi\ra$,
define its application on an operator $O$ such that expectation values of 
$O$ w.r.t. $T|\psi\ra$ agrees with that of $T(O)$ w.r.t. $|\psi\ra$. This motivates 
the following definition. 
\begin{definition}[operator transformation, invariance]
    The effect of transforming operator $O$ by $T$ is 
    $T(O)=T^\dag O T=T^{-1}OT$. It follows that $O$ is invariant under $T$ if $[O, T]=0$.
    \begin{equation}
        \la \psi|T^\dag OT|\psi\ra = \la \psi|T(O)|\psi\ra \implies T(O)=T^\dag OT
    \end{equation}
\end{definition}

\subsection{Observables and transformations}

Recall that $p = -i\hb \pd x$. Assuming that $\psi(x)$ may be expanded as a power-series.  
\begin{equation}\begin{aligned}\label{eqn:momentum as exponential}
    T(a)\psi(x)&=\psi(x-a)=\sum_{n=0}^\infty \df{(-a)^n}{n!}\pd x^n\psi(x)  \\ 
                &= \sum \df 1 {n!} \left(\df{-ia}{\hb}(-ih\pd x)\right)^n \psi(x)
\end{aligned}\end{equation}

\begin{definition}[exponential of an operator]
    We define the exponential of a Hermitian operator as a power-series, 
    with multiplication denoting composition
    \begin{equation}
        \exp\left(\alpha O\right) = \sum_{n=0}^\infty \df 1 {n!} (\alpha O)^n
    \end{equation}
    This definition is made precise by the formulation of Lie group and Lie algebra, 
    its main property of interest is 
    \begin{equation}
        \pd \alpha \exp(\alpha O) = \alpha \exp(\alpha O)
    \end{equation}
\end{definition}
\begin{definition}[observables generate transformation]
    We say that a Hermitian observable $O$ generates a one-parameter family of 
    transformations $T(a), a\in \R$ if 
    \begin{equation} 
        T(a)=\exp\left(-\df {i a}\hb O\right)
    \end{equation}
\end{definition}
Equation~\ref{eqn:momentum as exponential} shows that momentum generates axial translation. 

More generally, consider a (partial) basis $\{|o\ra\}$ corresponding to the eigenstates of an observable $O$ 
and let $\psi(o)=\la \psi|o\ra$. The basis is partial in the sense that it is the basis of 
one of possibly many components of the tensor product Hilbert space (e.g. position and spin). 
Note that $\psi(o)$ may be a state!
Define the effect of the unitary transformation $T_o(a)$ by $|o\ra \mapsto |o-a\ra$. 
Assuming that $\psi(o)$ may be expanded as a power series in terms of $o$. 
\begin{equation}
    \psi(o-a)=T_o(a)\psi(o)=\sum_{n=0}^\infty \df 1 {n!} ((-a)\, \pd o)^n \psi(o)
    =\exp\left(-a\,\pd o\right)\psi(o)
\end{equation}
Every unitary $T(a)$ is the complex exponential of a Hermitian $O$ which provides the 
phase factors for the eigenvalues of the unitary. We also have the freedom to choose $O$ 
up to a phase scaling factor $\alpha$ with both free value and unit: 
\begin{equation}\label{eqn:transformation_exponential}
    T(a)=\exp\left(\df i \alpha Q(a)\right)=\exp\left(-a\, \pd o\right)\implies Q(a) 
    = a Q = a (i\alpha \, \pd o)
\end{equation}

\begin{remark}
    Equation~\ref{eqn:hermitian_exponential} provides another perspective: 
    each Hermitian $Q$ defines an infinitesimal transformation $\left(1 + ia Q/N\right)$ when $N\to \infty$. 
    This operator is unitary up to a quadratic order:
    \begin{equation}\label{eqn:hermitian_exponential}
        \left(1 + ia Q/N\right)^\dag \left(1 + ia Q/N\right)
        =1+O(1/N^2)
    \end{equation}
    The cumulative action of applying $\left(1 + ia O/N\right)$ for $N$ times is $\exp(ia Q)$
\end{remark}
Returning to equation~\ref{eqn:transformation_exponential}, 
$\pd o$ effects the change $|o\ra\mapsto |o+do\ra$, so we choose $\alpha = -1/\hb$.
\begin{theorem}[Hamiltonian generates time translation] $\exp\left(-i H t/\hb\right)|t_0\ra = |t_0-t\ra$

    \textit{Proof:} this statement is equivalent to Schrodinger's equation. 
    \[ 
        \pd t|\psi\ra = -\df i \hb H|\psi\ra
    \] 
\end{theorem}

\begin{definition}[complete set of compatible observables (CSCO)]
    \label{def:csco}
    A set of Hermitian observables $\{O_j\}$ over a Hilbert space is 
    compatible if they pairwise commute, and complete if for every 
    set of eigenspaces $\{\mathcal S_{O_j, \lambda_k}\}$, their intersection 
    has dimension at most $1$. 
\end{definition}
\begin{example}[CSCO for central potential]
    For any Hamiltonian $H$ which commutes with $L$, the set $H, L_z, L^2$ is 
    a complete set of compatible observables. 
    They pairwise commute and $E_n, l, m$ uniquely determine an eigenstate. 
\end{example}



\subsection{Symmetry and conservation}
The key idea of this section is the diagram below. 
\begin{center}\begin{tikzcd}[row sep=huge, column sep=10em]
    \text{Transformation } T 
    \arrow[r, leftrightarrow, "T=\exp(-iO\lambda/\hbar)"]
    \arrow[d, leftrightarrow, swap, "{[H, T(a)]=0}"]
    & \text{Observable $O$} 
    \arrow[d, leftrightarrow, "{d_t\Pr[O=o, t]=0}"]
    \\
    \text{Symmetry} \arrow[r, leftrightarrow] & \text{Conservation}
\end{tikzcd}\end{center}


A system is defined by its time-evolution, which is defined by its Hamiltonian. 
We first establish the connection between transformation and symmetry. 
\begin{definition}[continuous symmetry]
    A system with Hamiltonian $H$ has continuous symmetry with respect to a family 
    of transformations $T(a)$, parameterized by $a\in \R$, if 
    \[ 
        \forall a\in \R, [H, T(a)] = 0
    \] 
\end{definition}

Transformations biject with observables, up to a constant factor. 
We show that continuous symmetry is equivalent to 
the conservation of an observable. 
\begin{theorem}[conservation is equivalent to commutativity] 
    The probability distribution for observating different values of a Hermitian $O$ 
    is conserved if and only if $[H, O] = 0$.

    \textit{Proof:} suppose $[H, O]=0$, then the energy eigenstates $|n\ra$ are also eigenstates of 
    $O$.
    \[ 
        |\psi_0\ra = \sum c_n |n\ra\implies \Pr[O=o_n, t=0] = |c_n|^2
    \] 
    Under time evolution, the probability distribution is unchanged 
    \[ 
        |\psi_t\ra = \sum \exp\left(iE_n t/\hb\right) c_n |n\ra\implies 
        \Pr[O=o_n, t] = |\exp\left(iE_n t/\hb\right) c_n|^2=|c_n|^2
    \]
    Conversely, distribution conservation in particular implies $\la O\ra = 0$ (Ehrenfest's theorem):
    \[ \begin{aligned}
        d_t\la O\ra 
            &= d_t\ang{\psi|O|\psi} \\ 
            &= \left(d_t|\psi\ra\right)^\dag O|\psi\ra + \left(O|\psi\ra\right)^\dag d_t|\psi\ra + \pd t O \\ 
            &= \df i \hb \ang{[H, O]}
    \end{aligned}\] 
    The expectation value vanishes for all $|\psi\ra$, so $[H, O]=0$. 
\end{theorem}

\begin{theorem}[continuous symmetry$\iff$conservation] \label{thm:conservation symmetry equivalence}
    For a Hermitian $O$, the probability distribution for observating different values of $O$ 
    is conserved if and only $H$ has continuous symmetry with respect to the family of transformations 
    generated by $O$
    \[ 
        T(a)=\exp(-ia O /\hb)
    \] 
    \textit{Proof:} $[H, \exp\left(-i aO/\hb\right)]\iff [H, O]=0$. Think in the eigenbasis throughout. 
\end{theorem}
\begin{remark}
    Theorem~\ref{thm:conservation symmetry equivalence} emphasizes the fundamental stochasticity 
    of quantum mechanics. The true quantum-mechanical counterpart of classical values are not the 
    measured values (which is not conserved), but its distribution. 
\end{remark}



\subsection{Selection rules}
Recall that commutativity ensures a simultaneous set of eigenvectors. This constitutes the most 
basic idea of a selection rule. We recall a result from linear algebra: 
\begin{theorem}[commutativity is equivalent to shared eigenbasis]
    two normal operators $A, B$ commute if and only if they share an eigenbasis. 

    \textit{Proof:} Let $v$ be an eigenvector of $A$ with eigenvalue $\lambda$, then 
    \[ 
        ABv = BAv = \lambda B v
    \] 
    This means $Bv$ is in the eigenspace of $A$ with eigenvalue $\lambda$. 
    Then $B$ maps eigenspaces of $A$ onto themselves. 
    Apply the spectral theorem in each eigenspace. 
\end{theorem}

Selection rules constrain the matrix elements of an operator based on commutativity. 
\begin{theorem}[Laporte's rule]
    Matrix elements of an operator that is odd under parity is nonzero only 
    for states with the same parity. Those for operators invariant under parity 
    is nonzero only for states with different parity. 
    
    \prf Consider an operator $O$ such that $\{O, \Pi\}=0$ and states $|n\ra, |m\ra$ with 
    parity 
    \[ 
        \Pi |n\ra = p_n|n\ra, \quad \Pi |m\ra = p_m|m\ra, \quad p_n, p_m\in \{-1, 1\} 
    \] 
    Consider the matrix element 
    \[ \begin{aligned}
        \ang{n|O|m} &= -\ang{n|\Pi^\dag O \Pi|m} \\ 
            &= -p_mp_n \ang{n|O|m}
    \end{aligned}\] 
    Rearranging the equation $\ang{n|O|m}(1+p_mp_n) = 0$ implies that states with the same parity 
    ($p_mp_n=1$) have vanishing matrix elements. The argument proceeds similarly for $[O, \Pi] = 0$. 
\end{theorem}
\begin{remark}
    One informal way to remember Laporte's rule is that the matrix elements of $O$ give the 
    transition matrix when $O$ happens to be the Hamiltonian. For a system which 
    cares about parity, only states which have the same parity can evolve into each other. 
\end{remark}
\begin{definition}[vector operator]
    Let $R_n(\theta)$ correspond to the transformation of rotating about $n$ by $\theta$, 
    it has a $3\times 3$ matrix representation $D_n(\theta)$. A 3-component vector 
    operator $V$ is a vector operator if for all $n, \theta$
    \[ 
        R_n^\dag(\theta) V R_n(\theta) = D_n(\theta)V
    \] 
    In other words, the operator transforms spatially. This is equivalent to if $V$ satisfies
    \[ 
        [L_i, V_j] = i\hb \epsilon_{ijk} V_k
    \] 
\end{definition}
Examples of three such vectors are $r, p, L$. 
\begin{theorem}[rotational selection rules for scalars] 
    A rank-$1$ operator $f$ satisfying 
    \[ [L_z, f] = [L_\pm, f] = [L^2, f] = 0\]
    has matrix elements of the form 
    \[ 
        \ang{n', l', m'|f|n, l, m} = \delta_{nn'}\delta_{ll'}\ang{n',l\, \|\, f\, \|\, n, l}
    \] 
\end{theorem}