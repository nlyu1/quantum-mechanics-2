\section{Second Quantization}
This section uses Chapter 2 of~\cite{altland2010condensed} 
available \href{https://www.tcm.phy.cam.ac.uk/~bds10/tp3/secqu.pdf}{here} and 
these \href{https://ethz.ch/content/dam/ethz/special-interest/phys/theoretical-physics/cmtm-dam/documents/qg/Chapter_05-06.pdf} 
{lecture notes} from ETH. 

Particles belong to one of two indistinguishable classes: fermions or bosons, which are 
antisymmetric and symmetric under particle exchange, respectively. Instead of explicitly 
symmetrizing a wave function, we can instead adapt our mathematical formulation to 
inherently account for indistinguishability and even varying number of particles. 

\begin{definition}[occupation number state]
    The occupation number state $|n_1, \cdots\ra$ denotes $n_i$ fermion or bosons 
    in level $i$. The total number of particles is given by 
    \[ 
        N = \sum n_i 
    \] 
    The space of all occupation number states for all $N$ is called Fock space. 
    Explicitly, let $S_\pm$ denote the (anti)-symmetrizer 
    \[ 
        S_\pm |i_1, \cdots, i_N\ra = \df 1 {N!} \sum_{P\in S_N} \sgn(P)\,  P|i_1, \cdots, i_N\ra 
    \] 
    the occupation state is $|n_1, \cdots\ra_{\pm} = S_\pm |i_1, \cdots, i_N\ra$ 
\end{definition}
\begin{proposition}
    $\la n_1, \cdots|n'_1\cdots\ra = \prod \delta_{n_i, n'_i}, \quad 
    \sum_{n_1, \cdots}|n_1, \cdots\ra \la n_1, \cdots| = 1$
\end{proposition}
\begin{definition}[creation and annihilation operators]
    The creation and annihilation operators $a_i^\dag, a_i$ of a particle 
    in level $i$ is defined to be 
    \malign{
        a_i^\dag |\cdots, n_i, \cdots\ra = \sqrt{n_i+1}|\cdots, n_i+1, \cdots\ra \\ 
        a_i|\cdots, n_i+1, \cdots\ra = \sqrt{n_i+1}|\cdots, n_i, \cdots\ra 
    }
    the second equation follows from the first by the adjoint definition. 
\end{definition}
\begin{theorem}[commutation relations]
    Consistent symmetrization implies 
    \begin{flalign*}
        && [a_i, a_j] &= [a_i^\dag, a_j^\dag] = 0, \quad [a_i, a_j^\dag] = \delta_{ij}I && \text{(bosons)} \\ 
        && \{a_i, a_j\} &= \{a_i^\dag, a_j^\dag\} = 0, \quad \{a_i, a_j^\dag\} = \delta_{ij}I && \text{(fermions)}
    \end{flalign*}
    They are called the canonical commutation relations (CCRs). 

    \prf we consider the fermions first. Given $|n_1, \cdots\ra=S_-|i_1, \cdots, i_N\ra$, we have 
    \malign{
        a_k^\dag a_j^\dag |n_1, \cdots\ra &= S_-|i_1, \cdots, i_N, j, k\ra  \\ 
        a_j^\dag a_k^\dag |n_1, \cdots\ra &= S_-|i_1, \cdots, i_N, k, j\ra 
    }
    Now they differ by a sign since $|i_1, \cdots, i_N, j, k\ra, |i_1, \cdots, i_N, k, j\ra$ 
    differ by a transposition $(N+1, N_2)$, so $\{a_i^\dag, a_j^\dag\}=0$. 
    Further note that antisymmetrization annihilates a vector if it contains any duplicate 
    basis, so $n_i=0, 1$. We omit the rest of the proof. 
\end{theorem}
\textbf{todo}
\begin{corollary}
    For fermions, the CCRs imply 
    \begin{flalign*}
        && a_j^2 = (a_j^\dag)^2 = 0,& \quad a_ja_j^\dag = I - a_j^\dag a_j \\ 
        && a_ja_k^\dag = -a_k^\dag a_j,& \quad a_ja_k = -a_ka_j && j\neq k
    \end{flalign*}
    For bosons, 
    \begin{flalign*}
        a_ja_j^\dag = I + a_j^\dag a_j, \quad a_ja_k^\dag = a_k^\dag a_j, \quad a_ja_k = a_ka_j
    \end{flalign*}
\end{corollary}
Let $V$ be the Hilbert space of one particle, then the occupation (Fock) space 
is $\mca F \cong V^0\oplus V^1\oplus V^2\oplus \cdots$. 
\begin{definition}[vacuum and general states]
    The vacuum state, denoted $|0\ra$ or $|\Omega\ra$, is the state annihilated by all annihilation operators. 
    One can also explicitly write out the occupation number state, for $\zeta=\pm 1$ denoting bosons 
    and fermions, respectively 
    \[ 
        |(n_i)\ra = \df 1 {\sqrt{\prod (n_i!)}} \left[\prod (a_i^\dag)^{n_i}\right]|\Omega\ra 
    \] 
\end{definition}
\begin{corollary}
    Pauli exclusion principle: for fermions $(a_i^\dag)^2=0$.  
    No single particle state can be occupied by more than one fermion. 
\end{corollary}
\begin{definition}[occupation number operator]
    the occupation number operator is the Hermitian operator defined by 
    \[ 
        N_i = a_i^\dag a_i, \quad N_i|(n_i)\ra = n_i |(n_i)\ra 
    \] 
    For both fermions and bosons, it satisfies 
    \[ 
        N_j (a_j^\dag)^n|\Omega\ra = n (a_j^\dag)^n|\Omega\ra, \quad [N_j, a_{k\neq j}^{(\dag)}]=0
    \] 
    The total particle number operator is 
    \[ 
        N = \sum a_i^\dag a_i 
    \] 
\end{definition}
Note that the level index $i$ implicitly assumes a basis (e.g. position or momentum, 
or spin along a particular direction). Suppose we wish to change into another basis $j$. 
\begin{proposition}
    change of basis formula $a^\dag_j = \la i|j\ra a_i^\dag, a_j = \la j|i\ra a_i$. 

    \prf note that single kets $|i\ra = a_i^\dag |\Omega\ra$. Then $|j\ra = a_j^\dag|\Omega\ra $, then 
    \[ 
        |j\ra = \la i|j\ra |i\ra = \la i|j\ra (a_i^\dag |\Omega) = a_j^\dag |\Omega\ra 
    \] 
\end{proposition}
\begin{proposition} the second quantization of a single-particle operator $O\in \End(V)$ is 
    \[ 
        O = \la i|O|j\ra a_i^\dag a_j 
    \] 
    
    \prf 
    Suppose $O$ is a single-particle operator with eigenbasis $o_\lambda, |\lambda\ra$. 
    In the second-quantization scheme, it has the representation 
    \[ 
        O = o_\lambda n_\lambda 
    \] 
    In other words, it counts how many particles are in each of the $|\lambda\ra$ eigenstates and 
    assigns the proper weights to them. Using a change of basis into arbitrary basis indexed by $i$ 
    or $j$ 
    \[ 
        o_\lambda n_\lambda 
        = \la \lambda|O|\lambda\ra a_\lambda^\dag a_\lambda 
        = \la \lambda|O|\lambda\ra \la i|\lambda\ra \la \lambda |j\ra a_i^\dag a_j \\ 
        = \la i|O|j\ra a_i^\dag a_j 
    \] 
\end{proposition}
We next consider the second quantization of two-body interaction operators. 
\begin{proposition}
    $V = \la i, j|V|k, l\ra a_i^\dag a_j^\dag a_ka_l$ 
\end{proposition}
\begin{definition}[field operators]
    Let $|\br\ra$ denote a particle localized at $\br$. This constitutes a special 
    basis. Given creation and annihilation operators in some basis indexed by $i$, denote 
    \[ 
        \la \br|i\ra = \phi_i(\br)
    \] 
    the creation and annihilation operators in the position basis, are called 
    the field operators 
    \[ 
        \hat \psi^\dag(r) = \phi_i^*(\br)a_i^\dag 
    \] 
\end{definition}