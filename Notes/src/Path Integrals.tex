\section{Path Integrals}
\subsection{Gaussian Integrals}
A single-variable Gaussian integral may be evaluated by a change into spherical coordinates: 
\begin{equation}
    G(a)=\intf dx\, \exp(-ax^2)=\sqrt{\df \pi a}
\end{equation}

More generally, we may consider a general Gaussian integral for a positive definite $n\times n$ matrix $A$
and offset $\omega$. 
\begin{equation}\label{eqn:real_multivariable_gaussian}
    \intf (dx_1\cdots dx_n)\, \exp\left(-x^TAx + \omega^Tx\right)=
    \exp\left(\df 1 4 \omega^TA^{-1}\omega\right)\sqrt{\df {\pi^N}{\det A}}
\end{equation}
The integral is evaluated by diagonalizing $A$ and completing the square, as below:
\[x^TAx - \omega^Tx = \left(x - \df{A^{-1}\omega}2\right)^TA\left(x-\df{A^{-1}\omega}2\right) 
- \df 1 4 \omega^TA^{-1}\omega\]

We may also consider integrating imaginary, oscillating exponentials: 
\begin{itemize}
    \item Put in by hand a convergence factor \(\exp(-\delta x^2)\), then take $\delta\to 0$
    \item Perform the change of variables $x'=\sqrt{-i\alpha} x$
\end{itemize}
The following integrals from \href{https://en.wikipedia.org/wiki/Common_integrals_in_quantum_field_theory}{Wikipedia}
are helpful:
\begin{equation}\label{eqn:complex_multivariable_gaussian}
\begin{aligned}
    \intf d^nx\, \exp\left(-\df 1 2x^T A x + \omega^Tx \right) &= \sqrt{\df{(2\pi)^n}{\det A}}\exp
        \left(\df 1 2 \omega^T A^{-1} \omega\right)
    \\ 
    \intf d^nx\, \exp\left(-\df 1 2x^T A x + i\omega^Tx \right) &= \sqrt{\df{(2\pi)^n}{\det A}}\exp
    \left(-\df 1 2 \omega^T A^{-1} \omega\right)
    \\ 
    \intf d^nx\, \exp\left(-\df i 2x^T A x + i\omega^Tx \right) &= \sqrt{\df{(2\pi i)^n}{\det A}}\exp
    \left(-\df i 2 \omega^T A^{-1} \omega\right)
\end{aligned}
\end{equation}

\subsection{Propagator}
Recall that in Lagrangian mechanics, the Lagrangian \(\mathcal L(q, \dot q, t)\) is usually 
associated with \(T-V\). Based on the Lagrangian, one may define the action functional as below, where \(t_1, t_2\) 
may be omitted if clear from context. 
\[S_{t_1, t_2}[q]=\int_{t_a}^{t_b} \mathcal L(q(t), \dot q(t), t)\]
\begin{example}[Classical free particle]
    Recall a free point particle with 
    \[\mathcal L = m\dot x^2/2\]
    The Lagrangian e.o.m and the resulting classical path are
    \[m\ddot x = 0\implies x_{\mathrm{cl}}(t)=\df{x_b-x_a}{t_b-t_a}(x-t_a)+x_a\]
    The corresponding classical action is 
    \[S[x_{\mathrm{cl}}] = 
        \int_{t_a}^{t_b} dt\, \df 1 2 m \left(\df{x_b-x_a}{t_b-t_a}\right)^2
        = \df{m(x_b-x_a)^2}{2(t_b-t_a)}\]
\end{example}

The theorem below shows how the path integral formulation quantum mechanics is based 
classical Lagrangian mechanics. The following theorem demonstrates its relation to 
the Hamiltonian formulation of quantum mechanics.

\begin{definition}[propagator]
    Given the Lagrangian for a system, its \textbf{propagator}, or \textbf{kernel}, is 
    the following integral over all paths \(q\) such that \(q(t_a)=x_a, q(t_b)=x_b\)
    \begin{equation}\label{eqn:def_propagator}
        K(x_b, t_b, x_a, t_a)=\int D[q]\, \exp\left(\df i \hbar S[q]\right)
        =\int D[q]\, \exp\left(\df i \hbar \int_{t_a}^{t_b}dt\, \mathcal L(q(t), \dot q(t), t)\right)
    \end{equation}
\end{definition}
    
\begin{theorem}[path integral formulation]
    The kernel satisfies \[K(x_b, t_b, x_a, t_a)=\ang{x_b, t_b|x_a, t_a}\]
    In other words, the Hamiltonian and Lagrangian formulation of quantum mechanics are 
    compatible in the following way (assuming a time-indepent Hamiltonian)
    \begin{equation}\label{eqn:propagator_hamiltonian}
        \int D[q]\, \exp\left(\df i \hbar \int_{t_a}^{t_b}dt\, \mathcal L(q(t), \dot q(t), t)\right)
        = \ang{x_b\left|\exp\left(-\df i \hb (t_b-t_a)H\right)\right|x_a} 
    \end{equation}
\end{theorem}

\begin{remark} 
    Several remarks are in order:
    \begin{itemize}
    \item 
        Every path $q$ contributes a factor with absolute value $1$: contributions only differ in phase. 
        Conceptually, the first-order endpoint-preserving variations vanish at the classical path (which satifies the 
        Lagrangian equations of motion), resulting in in-phase contributions, while those paths far from 
        the classical one are easily out of phase.
    \item 
        The destructive interference far away from the classical path relies on the large value of $\Delta S / \hb$. 
        This qualifies how quantum mechanics degenerates to classical mechanics in the limit $\hb\to 0$, or when action 
        scales are large when compared to $\hb$, as in the case for most macroscopic systems. 
    \item 
        Should we choose to describe our system as an evolving field $\Psi(x, t)$ instead of a path $q(t)$,
        the path-integral formulation generalizes in a straightforward way:
        \[\int D[\Psi]\exp\left(\df i \hb \int d^4x\, \mathcal L(\Psi, \dot \Psi, t)\right)\]
    \item 
        We're assuming spinless particles. 
        There are subtle ties to commutativity here. 
    \item 
        The position basis in $K(b, a)=\la x_b|U(t_b-t_a)|x_a\ra$ is special. We cannot use the Lagrangian 
        should we choose a different basis for the propagator.  
    \end{itemize}
\end{remark}

The propagator $K$ determines the dynamics. We take the basis $|x, t\ra$ 
and view the system state as defined on the whole space-time, with time evolution via varying $t$: 
\begin{equation}\begin{aligned}\label{eqn:propagator_evolution}
    \psi(x, t) 
        &= \ang{\psi|x, t} = \la \psi|{\sum}_{x'}\la x, t|x', t_0\ra |x', t_0\ra  \\ 
        &= {\sum}_{x'}K(x, t, x', t_0)\la \psi|x', t_0\ra \\ 
        &= {\sum}_{x'}K(x, t, x', t_0)\psi(x', t_0)
\end{aligned}\end{equation}
Alternatively, assuming a time-invariant Hamiltonian and an energy basis $\{|n, t\ra\}$
\begin{equation}\begin{aligned}\label{eqn:propagator_energy_eigenstates}
    K(x, t, x', t') 
        &= \la x, t|x', t'\ra \\ 
        &= {\sum}_n \la x, t'|n\ra \la n|x, t\ra \\ 
        &= {\sum}_n \exp\left(-\df i \hb E_n(t-t')\right)\psi_n^*(x)\psi_n(x')
\end{aligned}\end{equation}
The propagator satisfies the Schrodinger equation: fixing $x', t'$ (recall equation~\ref{eqn:propagator_hamiltonian})
\begin{equation}
    i\hb \pd t K(x, t, x', t') = H_t K(x, t, x', t')
\end{equation}
\begin{remark}
    The dynamics of the system is captured by a one-parameter family of unitaries $U_t$, 
    obtained by integrating the exponential of the Hamiltonian (or computing the Schrodinger 
    equation), which send initial states $|\psi_0\ra$ to $|\psi_t\ra$. 
    The propagator is simply the representation of this unitary in the position basis. 
    To see why, recall the action of a linear operator $A$ with matrix representation $(A_{ij})$ 
    in a finite-dimensional Hilbert space 
    \[ 
        (Av)_i = \sum_j A_{ij} v_j 
    \] 
    The matrix element representation $A_{ij}$ may be viewed instead as a map $A(i, j)=A_{ij}$ 
    and the states similar maps $v(j) = v_j$. 
    \[ 
        (Av)(i) = \sum_j A(i, j)v(j)
    \] 
    The direct analogue of this in infinite-dimensional Hilbert space is to replace summing 
    over the discrete index $j$ with integration over a continuous variable $x$ 
    \[ 
        \psi_t(x) = (U_t\, \psi_0)(x) = \int K_{0\to t}(x, x')\psi_0(x')\, dx' 
    \] 
\end{remark}

\subsection{Integration over paths}
Consider the following method of parameterizing all possible paths $(x_a, t_a)\to (x_b, t_b)$. 
Discretize time by units of $\epsilon$: let $t_0=t_a, t_N=t_b, N\epsilon = t_b-t_a$, and $t_{j+1}=t_j+\epsilon$. 
\begin{equation}
    K(b, a)\sim \lim_{\epsilon \to 0, N\epsilon = \Delta t}\int\cdots \int (dx_1\cdots dx_{N-1}) \exp\left(\df i \hb S[q(x_1\cdots x_{N-1})]\right)
\end{equation}
Note that $x_0, x_N$ are fixed, so they are not variables to be integrated. 
Under this parameterization, the values $\{x_1\cdots x_{N-1}\}$ parameterizes a path $q$ under the following substitution:
\begin{equation}\begin{aligned}
    S[q]&=\int_{t_a}^{t_b}\mathcal L(q, \dot q, t)\, dt = \sum_{j=0}^{N-1}\mathcal L\left(x_i, \df{x_{i+1}-x_i}{\epsilon}, t\right)
\end{aligned}\end{equation}

\subsection{Separable Lagrangian}
Let $q_{\mathrm{cl}}$ denote the classical path satisfying Euler-Lagrange equations. 
Consider the following quantity for an endpoint-preserving perturbation $\eta$ and time-invariant Lagrangian: 
\begin{equation}
\begin{aligned}
    \mathcal L(q_{\mathrm{cl}}+\epsilon \eta, \dot q_{\mathrm{cl}}+\epsilon \dot \eta)  
    &= \sum_{n=1}^\infty \df 1 {n!} \left[\epsilon \left(\eta \pd q + \dot \eta \pd{\dot q}\right)\right]^n \mathcal L(q_{\mathrm{cl}}, \dot q_{\mathrm{cl}}, t) \\ 
    &= \left[1 + 
        \epsilon (\eta \pd q + \dot \eta \pd{\dot q}) + 
        \epsilon^2\left(\eta^2 \pd{q}^2 + 2(\eta \pd q)(\dot \eta d_{\dot q}) + \dot \eta^2 \pd{\dot q}^2\right)
        +O(\epsilon^3)\right]\mathcal L_{\mathrm{cl}}
\end{aligned} 
\end{equation}
The zeroth-order term denotes the classical action, the first-order term vanishes by the 
Euler-Lagrange equations, and the second-order terms preserve the quadratic coefficients. 

\begin{theorem}[quadratic Lagrangians are separable]
    For quadratic Lagrangians of the form 
    \[\mathcal L(q, \dot q, t)=Aq^2+B\dot q^2\]
    Both the Lagrangian and the action separate linearly by classical paths 
    \begin{equation}\begin{aligned}
        \mathcal L(q_{\mathrm{cl}} + \eta, \dot q_{\mathrm{cl}} + \dot \eta) &= 
            \mathcal L(q_{\mathrm{cl}}, \dot q_{\mathrm{cl}}) + \mathcal L(\eta, \dot \eta) \\ 
        S[q_{\mathrm{cl}}+\eta] &= S[q_{\mathrm{cl}}] + S[\eta]
    \end{aligned}\end{equation}
\end{theorem}

\begin{theorem}[propagator for separable Lagrangian]
    When the Lagrangian separates, the propagator in equation~\ref{eqn:def_propagator} takes the following form. 
    The path integral over $\eta$ is agnostic towards $x_a, x_b$ and only depends on $t_b-t_a$ since the Lagrangian 
    is time-invariant. 
    \[\begin{aligned}\label{eqn:separable_propagator}
        K(x_b, t_b, x_a, t_a)&=\int D[q]\, \exp\left(\df i \hbar S[q]\right) 
        =\int D[\eta]\, \exp\left(\df i \hbar S[q_{\mathrm{cl}}+\eta]\right) \\ 
        &= \exp\left(\df i \hb S_{\mathrm{cl}}\right) \int D[\eta]\, \exp\left(\df i \hbar S[\eta]\right)
        =A(t_b-t_a)\exp\left(\df i \hb S_{\mathrm{cl}}\right)
    \end{aligned}\]
\end{theorem}

\begin{example}[free particle]
    Recall a free particle has Lagrangian $\mathcal L(x, \dot x)=\df 1 2 m \dot x$. 
    \begin{equation}
        K(x, t, x', t') = \sqrt{\df m {2\pi i \hbar \Delta t}}\exp\left(\df{im(x-x')^2}{2\hb \Delta t}\right)
    \end{equation}
    Note the appearance of classical action in the spatially dependent exponential term. 
\end{example}

\begin{example}[simple harmonic oscillator]
    Consider the following Lagrangian 
    \[L=\dfrac {m \dot x^2}{2} - \dfrac{m \omega^2}{2}x^2\]
    The corresponding classical action is 
    \[
        S_{\mathrm{cl}}=\frac{m \omega}{2 \sin(\omega \Delta t)}   \left(\left(x_a^2+x_b^2\right) \cos (\omega  \Delta t)-2 x_a x_b\right)
    \]
    Recalling the simple Harmonic oscillator solution are all real and of the following form 
    \[\psi_n(x)=\dfrac 1 {\sqrt{2^nn!}}\left(\dfrac{m\omega}{\pi \hbar}\right)^{1/4}\exp\left(-\dfrac{m\omega x^2}{2\hbar}\right) 
        H_n \left(\sqrt{\dfrac{m\omega}{\hbar}x}\right)\]
    Perform this substitution and invoke equation~\ref{eqn:propagator_energy_eigenstates}
    \[
        \sqrt{\dfrac{m\omega}{\hbar}}x_a\mapsto \xi, \sqrt{\dfrac{m\omega}{\hbar}}x_b\mapsto \eta
    \]
    \[ 
        K=\sqrt{\dfrac{m\omega}{2\pi \hbar i \sin \omega \Delta t}}
            \exp \left(\dfrac 1 2 (\xi^2+\eta^2)-\dfrac{\xi^2+\eta^2 - 2\xi\eta \exp(-i\Delta t \omega)}{1-\exp(-2i\Delta t\omega)}\right)
    \]
    The exponential term is the exponential of the classical action. 
    \begin{equation}
        K(b, a)=\sqrt{\dfrac{m\omega}{2\pi \hbar i \sin \omega \Delta t}}\exp\left(\dfrac i \hbar S_\mathrm{cl}\right)
    \end{equation}
\end{example}

